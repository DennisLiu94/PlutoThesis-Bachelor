% !Mode:: "TeX:UTF-8" 

\BiChapter{绪论}{Introduction}

\BiSection{模板的使用方法介绍}{Introduction to the application method of the template}
如果您是~\LaTeX~新手,请您在使用此模板之前,先观看一下模板的使用说明~PPT 演示文档《哈工大学位论文~\LaTeX~模板使用方法介绍》,先大致了解一下此模板的使用方法,之后再准备使用此模板撰写学位论文;如果您有一定的~\LaTeX~技术基础,可以跳过此步骤。

\BiSection{哈尔滨工业大学~\LaTeX~技术交流~QQ 群介绍}{Introduction to the QQ groups for \LaTeX~technical exchange in HIT}
《哈工大硕博学位论文~\LaTeX~模板》项目现已加入哈工大研究生“学术桥”活动中,其两个官方~QQ 群分别为

\centerline{学术桥-\LaTeX~交流群~1:38872389}
\centerline{学术桥-\LaTeX~交流群~2:88984107}
\noindent 欢迎大家加入。加入~QQ 群之后,请大家将自己的名字前面加上当前月份标识,如果您没有标识,在~QQ 群人数已满但仍有人要加入此~QQ 群时,我们会将您优先请出~QQ 群,谢谢合作!您可以在~QQ 群中和其他人讨论关于此模板或其它~\LaTeX~技术相关的任何问题,在提问之前,可以先去~\href{http://bbs.ctex.org/}{CTEX 论坛}或其它网站搜索您所要得到的解决方案,然后确定是否要继续提问,从而节省您的宝贵时间。群共享中包含有大量的~\LaTeX~技术资料,方便大家下载阅读,同时还包含了最新的《哈工大硕博学位论文~\LaTeX~模板》。

原则上,模板在~Google Code 的~\href{http://code.google.com/p/plutothesis/}{PlutoThesis 项目}的\href{http://code.google.com/p/plutothesis/downloads/list}{下载列表}中也进行了同步更新,但是模板维护人员不再解答用户在此网站提出的问题,如有问题请加入上述的两个~QQ 群中再询问,敬请谅解。

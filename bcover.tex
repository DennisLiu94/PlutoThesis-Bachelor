% !Mode:: "TeX:UTF-8" 

\newcommand{\chinesethesistitle}{哈尔滨哈飞集团汽车博物馆设计} %授权书用,无需断行
\newcommand{\englishthesistitle}{\uppercase{Hehe Hehe Hehe Hehe Hehe Hehe Hehe }} %\uppercase作用:将英文标题字母全部大写;
\newcommand{\chinesethesistime}{2010~年~7~月}  %封面底部的日期中文形式


\ctitle{哈尔滨哈飞集团汽车博物馆设计}  %封面用论文标题,自己可手动断行
% 本科内封的两行 
\ctitleone{哈尔滨哈飞集团} %用于内封上的两行标题,请手动根据标题内容酌情断行
\ctitletwo{汽车博物馆设计} %用于内封上的两行标题,请手动根据标题内容酌情断行
\cdegree{\cxueke\cxuewei}
\csubject{计算机科学与技术}                 %(~按二级学科填写~)
\caffil{计算机科学与技术学院} %(在校生填所在系名称,同等学力人员填工作单位)
\cauthor{安~~~~娜}
\csupervisor{某某某教授} %导师名字
\cstuid{1100300101} %本科蛋疼学号
%\csateDate{2010年6月1日}


\cdate{\chinesethesistime}


\iffalse
\BiAppendixChapter{摘~~~~要}{}  %使用winedt编辑时文档结构图(toc)中为了显示摘要,故增加此句;
\fi
\cabstract{
摘要是论文内容的高度概括,应具有独立性和自含性,即不阅读论文的全文,就能获得必要的信息。
摘要应包括本论文的目的、主要研究内容、研究方法、创造性成果及其理论与实际意义。
摘要中不宜使用公式、化学结构式、图表和非公知公用的符号和术语,不标注引用文献编号。避免将摘要写成目录式的内容介绍。
}

\ckeywords{关键词~1;关键词~2;关键词~3;……;关键词~6(关键词总共~3~—~6~个,最后一个关键词后面没有标点符号)}

\eabstract{
Externally pressurized gas bearing has been widely used in the field of aviation, semiconductor, weave, and measurement apparatus because of its advantage of high accuracy, little friction, low heat distortion, long life-span, and no pollution. In this thesis, based on the domestic and overseas researching……

}

\ekeywords{keyword 1, keyword 2, keyword 3, ……, keyword 6 (no punctuation at the end) 英文摘要与中文摘要的内容应一致,在语法、用词上应准确无误。}

\makecover
\clearpage 